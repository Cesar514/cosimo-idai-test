\subsection{Notation and Occupancy-Grid Construction}
Let the maze have width $W$ and height $H$ in cell coordinates. The benchmark converts each maze to a binary occupancy lattice
\begin{equation}
G \in \{0,1\}^{R \times C}, \quad R = 2H + 1, \quad C = 2W + 1,
\end{equation}
where $G_{r,c}=0$ denotes free space and $G_{r,c}=1$ denotes blocked space.
The cell-to-lattice map used by the implementation is
\begin{equation}
\phi(x,y) = (2y+1,\,2x+1),
\end{equation}
so the benchmark start and goal nodes are
\begin{equation}
s=\phi(x_s,y_s), \qquad g=\phi(x_g,y_g).
\end{equation}

For a lattice node $n=(r,c)$, the neighborhood is either 4-connected or 8-connected:
\begin{equation}
\mathcal{N}_4(n)=\{(r-1,c),(r,c+1),(r+1,c),(r,c-1)\}\cap\Omega,
\end{equation}
\begin{equation}
\mathcal{N}_8(n)=\mathcal{N}_4(n)\cup\{(r-1,c-1),(r-1,c+1),(r+1,c+1),(r+1,c-1)\}\cap\Omega,
\end{equation}
with $\Omega$ the in-bounds lattice set.
The step cost in the baseline planners is
\begin{equation}
c(u,v)=
\begin{cases}
\sqrt{2}, & \text{if }u\rightarrow v\text{ is diagonal},\\
1, & \text{otherwise}.
\end{cases}
\end{equation}

Heuristic options are
\begin{equation}
h_{\mathrm{man}}(n,g)=|r-r_g|+|c-c_g|,
\end{equation}
\begin{equation}
h_{\mathrm{euc}}(n,g)=\sqrt{(r-r_g)^2+(c-c_g)^2},
\end{equation}
\begin{equation}
h_{\mathrm{cheb}}(n,g)=\max\{|r-r_g|,|c-c_g|\}.
\end{equation}

\subsection{Planner and Benchmark Objectives}
For a path $\pi=(n_0,\dots,n_K)$ with $n_0=s$ and $n_K=g$, the weighted path-cost objective is
\begin{equation}
J_{\mathrm{cost}}(\pi)=\sum_{k=1}^{K} c(n_{k-1},n_k).
\end{equation}

The baseline priority keys used in the benchmark are:
\begin{equation}
f_{\mathrm{A*}}(n)=g(n)+w\,h(n), \qquad w\ge 1,
\end{equation}
\begin{equation}
f_{\mathrm{Dijkstra}}(n)=g(n),
\end{equation}
\begin{equation}
f_{\mathrm{GBFS}}(n)=h(n),
\end{equation}
and BFS minimizes hop count
\begin{equation}
J_{\mathrm{hop}}(\pi)=K.
\end{equation}

For A*, the secondary tie key is
\begin{equation}
\tau(n)=
\begin{cases}
h(n), & \text{low\_h},\\
-g(n), & \text{high\_g},\\
0, & \text{fifo},
\end{cases}
\end{equation}
and the heap key is $(f(n),\tau(n),t(n))$ where $t(n)$ is a monotonically increasing insertion counter.

\paragraph*{Path validation and measured length}
Given returned path $P=(p_0,\dots,p_K)$, validity requires:
\begin{equation}
p_0=s,\quad p_K=g,\quad p_k\in\Omega,\quad G_{p_k}=0,\ \forall k,
\end{equation}
and for each segment, every rasterized Bresenham cell is also in-bounds and free.
If $\mathcal{B}(p_{k-1},p_k)$ denotes the rasterized segment cell sequence, the measured benchmark path length is
\begin{equation}
L(P)=\sum_{k=1}^{K}\left(\left|\mathcal{B}(p_{k-1},p_k)\right|-1\right).
\end{equation}

\paragraph*{Planner ranking objective}
For planner $p$ over $M$ mazes, with success indicator $I_{p,m}\in\{0,1\}$, solve time $t_{p,m}$, measured length $L_{p,m}$, and expansions $e_{p,m}$:
\begin{equation}
S_p=\frac{1}{M}\sum_{m=1}^{M} I_{p,m},\qquad
\mathcal{M}_p^+=\{m\mid I_{p,m}=1\}.
\end{equation}
Define the shared-success set
\begin{equation}
\mathcal{M}_{\mathrm{shared}}=\{m\mid I_{q,m}=1,\ \forall q\in\mathcal{P}\}.
\end{equation}
The comparable means used for ranking are
\begin{equation}
\bar{t}_p^{c}=
\begin{cases}
\frac{1}{|\mathcal{M}_{\mathrm{shared}}|}\sum_{m\in\mathcal{M}_{\mathrm{shared}}} t_{p,m}, & |\mathcal{M}_{\mathrm{shared}}|>0,\\
\frac{1}{M}\sum_{m=1}^{M} t_{p,m}, & \text{otherwise},
\end{cases}
\end{equation}
\begin{equation}
\bar{L}_p^{c}=
\begin{cases}
\frac{1}{|\mathcal{M}_{\mathrm{shared}}|}\sum_{m\in\mathcal{M}_{\mathrm{shared}}} L_{p,m}, & |\mathcal{M}_{\mathrm{shared}}|>0,\\
\frac{1}{|\mathcal{M}_p^{+}|}\sum_{m\in\mathcal{M}_p^{+}} L_{p,m}, & \text{otherwise}.
\end{cases}
\end{equation}
With
\begin{equation}
\bar{e}_p=\frac{1}{|\mathcal{M}_p^{+}|}\sum_{m\in\mathcal{M}_p^{+}} e_{p,m}, \qquad
\bar{t}_p=\frac{1}{M}\sum_{m=1}^{M} t_{p,m},
\end{equation}
ranking is lexicographic on
\begin{equation}
\left(-S_p,\ \bar{t}_p^{c},\ \bar{L}_p^{c},\ \bar{e}_p,\ \bar{t}_p,\ \text{name}_p\right).
\end{equation}

\subsection{Local Control Abstraction}
\paragraph*{Path-to-waypoint map}
The simulator converts planner outputs to world-frame waypoints.
If a path is recognized as cell-grid coordinates $(x_k,y_k)$:
\begin{equation}
w_k=\left((x_k+0.5)s_c,\ (y_k+0.5)s_c\right),
\end{equation}
where $s_c$ is \texttt{cell\_size}.
If recognized as occupancy-lattice coordinates:
\begin{equation}
w_k=\left(0.5\,s_c\,x_k,\ 0.5\,s_c\,y_k\right).
\end{equation}
Cell-grid waypoints are additionally compressed by removing collinear interior points.

\paragraph*{Waypoint tracking law}
At control step $t$, with robot pose $(x_t,y_t,\psi_t)$ and active waypoint $(x_j,y_j)$:
\begin{equation}
d_t=\sqrt{(x_j-x_t)^2+(y_j-y_t)^2},
\end{equation}
\begin{equation}
\theta_t=\operatorname{atan2}(y_j-y_t,\ x_j-x_t),\qquad
e_{\psi,t}=\mathrm{wrap}_{[-\pi,\pi]}(\theta_t-\psi_t).
\end{equation}

The angular command is the clipped proportional target
\begin{equation}
\omega_t^\star=\arg\min_{|\omega|\le \omega_{\max}}|\omega-k_\psi e_{\psi,t}|
=\mathrm{clip}(k_\psi e_{\psi,t},-\omega_{\max},\omega_{\max}).
\end{equation}

The linear target is the largest feasible speed under speed, distance, and braking bounds:
\begin{equation}
v_t^\star=\alpha_t\cdot \max_{v\ge 0} v
\quad\text{s.t.}\quad
v\le v_{\max},\ \ v\le d_t,\ \ v\le\sqrt{2a_{\mathrm{dec}}d_t},
\end{equation}
equivalently
\begin{equation}
v_t^\star=\alpha_t\cdot\min\left\{v_{\max},\ d_t,\ \sqrt{2a_{\mathrm{dec}}d_t}\right\}.
\end{equation}
The heading gate $\alpha_t\in[0,1]$ is
\begin{equation}
\alpha_t=
\begin{cases}
0, & |e_{\psi,t}|\ge \theta_{\mathrm{turn}},\\
1, & |e_{\psi,t}|\le \theta_{\mathrm{drive}},\\
1-\sigma\!\left(\dfrac{|e_{\psi,t}|-\theta_{\mathrm{drive}}}{\theta_{\mathrm{turn}}-\theta_{\mathrm{drive}}}\right), & \text{otherwise},
\end{cases}
\end{equation}
with smoothstep
\begin{equation}
\sigma(z)=z^2(3-2z), \quad z\in[0,1].
\end{equation}

Commands are passed through acceleration/deceleration slew limits with control step $\Delta t$:
\begin{equation}
u_t=u_{t-1}+\mathrm{clip}\!\left(u_t^\star-u_{t-1},-\Delta_u,\Delta_u\right),
\end{equation}
where
\begin{equation}
\Delta_u=
\begin{cases}
a_{u,+}\Delta t, & \operatorname{sgn}(u_t^\star)=\operatorname{sgn}(u_{t-1})\ \wedge\ |u_t^\star|>|u_{t-1}|,\\
a_{u,-}\Delta t, & \text{otherwise},
\end{cases}
\end{equation}
applied separately to $u=v$ and $u=\omega$.

Then the turn-rate-dependent linear cap is enforced:
\begin{equation}
\eta_t=\mathrm{clip}\!\left(\frac{|\omega_t|}{\omega_{\max}},0,1\right),\qquad
v_{\mathrm{turn},t}=v_{\max}\max\left(0.18,\ 1-0.68\,\sigma(\eta_t)\right),
\end{equation}
\begin{equation}
v_t\leftarrow \mathrm{clip}(v_t,-v_{\mathrm{turn},t},v_{\mathrm{turn},t}).
\end{equation}

For differential drive with axle track $b$ and wheel radius $r$:
\begin{equation}
v_{L,t}=v_t-\frac{b}{2}\omega_t,\qquad
v_{R,t}=v_t+\frac{b}{2}\omega_t,
\end{equation}
\begin{equation}
\Omega_{L,t}=v_{L,t}/r,\qquad \Omega_{R,t}=v_{R,t}/r.
\end{equation}
Waypoint index $j$ is advanced when $d_t\le \varepsilon_j$, where $\varepsilon_j$ is the per-waypoint tolerance (larger at the final waypoint).

\subsection{Complexity Statements}
Let $V=RC$ and $E\le 8V$ for the occupancy lattice graph.

\begin{equation}
T_{\mathrm{A*}/\mathrm{Dijkstra}/\mathrm{GBFS}}=O(E\log V),\qquad
M_{\mathrm{A*}/\mathrm{Dijkstra}/\mathrm{GBFS}}=O(V),
\end{equation}
because open lists are binary heaps and closed/g-score/came-from are hash maps/sets.

\begin{equation}
T_{\mathrm{BFS}}=O(V+E),\qquad M_{\mathrm{BFS}}=O(V),
\end{equation}
from deque frontier plus visited/parent maps.

For one returned path $P$:
\begin{equation}
T_{\mathrm{validate}}=O\!\left(\sum_{k=1}^{K}\left|\mathcal{B}(p_{k-1},p_k)\right|\right)
=O(L(P)).
\end{equation}

For $M$ mazes and $|\mathcal{P}|$ planners:
\begin{equation}
T_{\mathrm{benchmark}}=
O\!\left(
\sum_{m=1}^{M}\sum_{p\in\mathcal{P}}
\left(T_{\mathrm{plan}}(p,m)+T_{\mathrm{validate}}(p,m)\right)
\right).
\end{equation}

Per control update, command synthesis is constant-time, and actuator dispatch is linear in wheel-joint count $n_w$:
\begin{equation}
T_{\mathrm{control\ step}}=O(1+n_w).
\end{equation}
