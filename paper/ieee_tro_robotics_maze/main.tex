\documentclass[journal]{IEEEtran}

% Minimal, template-friendly package set
\usepackage{graphicx}
\usepackage{amsmath, amssymb}
\usepackage{booktabs}
\usepackage{url}

% Uncomment if your venue allows (IEEEtran generally works with hyperref, but check)
% \usepackage[hidelinks]{hyperref}

\begin{document}

\title{Deterministic Static Grid-Maze Planner Benchmarking for Reproducible Robotics Experiments}

\author{Cesar Contreras%
\thanks{Cesar Contreras is an independent robotics researcher based in California, USA.}%
}

\maketitle

\begin{abstract}
This paper presents a deterministic maze-navigation benchmark stack for reproducible robotics
experiments across multiple simulation backends. The implementation unifies planner execution,
path validation, benchmarking, and artifact generation in a single repository workflow built
around deterministic seeds and explicit benchmark exports. In the executed snapshot, the benchmark
evaluates 12 planners on 50 generated \(15\times15\) mazes with complete trial-level reporting
for success, runtime, path length, and expansions. All planners solve all trials; within this
static regime, weighted A* has the lowest mean solve time (0.35 ms), while search-effort profiles
show non-trivial runtime-expansion tradeoffs across methods. The pipeline also includes deterministic
simulation regression checks with fixed screenshot expectations to support repeatability audits.
The contribution is a repository-grounded, traceable benchmarking workflow for static grid-maze
navigation; claims are intentionally scoped to this benchmark regime and do not assert general
superiority in dynamic or kinodynamic navigation domains.
\end{abstract}

\begin{IEEEkeywords}
mobile robot navigation, deterministic benchmarking, path planning, reproducibility, simulation
\end{IEEEkeywords}

\section{Introduction}
Grid-based planning remains a practical foundation for robotic navigation in constrained environments, and recent surveys show that graph-search, heuristic, and hybrid variants remain competitive baselines in deployed systems \cite{S_nchez_Ib_ez_2021,Karur_2021,Xiao_2022,Liu_2023}. A persistent experimental gap is not planner availability, but comparable execution: simulator differences, heterogeneous planner return schemas, and weak path-validity checks can dominate reported deltas. This work targets that gap for one bounded domain: static occupancy-grid mazes.

The implemented stack links a command-level entrypoint (\texttt{scripts/sim\_runner.py}) to deterministic execution in \texttt{robotics\_maze/src/main.py}: argument parsing to \texttt{RunConfig}, per-episode seeding, module discovery for maze generator/planner/simulator, and standardized episode logging. The runtime is robust to machine variability by selecting PyBullet or MuJoCo when available, with deterministic fallback when dependencies are missing. This emphasis on portable execution is aligned with recent ROS2/cloud robotics infrastructure trends \cite{Baumann_2021,He_2022,Chen_2024,Shcherbyna_2025}.

The same repository provides a benchmark harness (\texttt{robotics\_maze/src/benchmark.py}) that normalizes planner outputs, validates geometric path feasibility against occupancy grids, and emits trial-level CSV/Markdown artifacts. In the tracked snapshot (\texttt{robotics\_maze/results/benchmark\_summary.md}), 12 planners are evaluated on 50 generated \(15\times15\) backtracker mazes.

This paper contributes three bounded, repository-grounded results:
\begin{itemize}
    \item \textbf{Infrastructure method contribution:} a deterministic planner-evaluation runtime that combines explicit seed propagation, heterogeneous planner I/O normalization, geometric path validation, and backend fallback behavior in one executable workflow.
    \item \textbf{Evaluation-protocol contribution:} a shared-success comparability policy with rotated planner order and deterministic lexicographic ranking by success rate, comparable solve time, mean expansions, and overall mean solve time. Path length is reported descriptively but excluded from ranking when any-angle outputs are present.
    \item \textbf{Evidence contribution:} an executed benchmark snapshot (12 planners, 50 mazes) with trial-level artifacts, plus deterministic simulation regression checks and screenshot outputs for reproducibility auditing.
\end{itemize}

Traceability artifacts (claims tables, figure manifests, citation audits) are included to support reproducibility and reviewability, but are not presented as standalone novelty claims.

The remainder of the manuscript details the implemented architecture and protocol, reports current benchmark evidence, and delineates what remains future work.

\section{Related Work}
Recent navigation literature continues to improve planner quality across classical, hybrid, and learned approaches \cite{S_nchez_Ib_ez_2021,Karur_2021,Xiao_2022,Liu_2023,Abdulsaheb_2023,Hu_2025}. At the same time, multiple surveys report that reproducibility bottlenecks often come from integration variance (runtime stack, simulator configuration, and evaluation protocol) rather than algorithmic novelty alone \cite{Hewawasam_2022,Loganathan_2023}.

For this reason, the closest prior art for our scope is benchmark/runtime infrastructure rather than a single planner family. Benchmark platforms such as Bench-MR \cite{Heiden_2021}, MotionBenchMaker \cite{Chamzas_2022}, and CoBRA \cite{Mayer_2024} focus on reusable benchmark construction and reporting. Runtime frameworks in ROS2/cloud settings emphasize deployment portability and orchestration across heterogeneous environments \cite{Baumann_2021,He_2022,Chen_2024,Shcherbyna_2025}. Our manuscript is positioned at their intersection for one bounded regime: deterministic static grid-maze planner comparison with repository-local traceability.

\begin{table*}[t]
\centering
\caption{Closest infrastructure-oriented prior work and the bounded incremental delta of this paper.}
\label{tab:closest_work_delta}
\footnotesize
\setlength{\tabcolsep}{4.2pt}
\begin{tabular}{p{0.19\textwidth} p{0.31\textwidth} p{0.43\textwidth}}
\toprule
Closest work & Reported primary scope & Incremental delta in this paper \\
\midrule
Bench-MR \cite{Heiden_2021} &
Motion-planning benchmark for wheeled mobile robots. &
Integration of repository-level deterministic seed propagation with trial-level geometric path validation. \\
\midrule
MotionBenchMaker \cite{Chamzas_2022} &
Tooling for generating and benchmarking motion-planning datasets. &
Standardized output normalization across heterogeneous planner schemas coupled with shared-success ranking. \\
\midrule
CoBRA \cite{Mayer_2024} &
Composable benchmark framework for robotics applications. &
Coupled execution stack for planner benchmarking with explicit simulator-fallback and visual regression. \\
\bottomrule
\end{tabular}
\end{table*}


Adjacent systems such as FogROS2-LS and Arena 4.0 provide important deployment/simulation context \cite{Chen_2024,Shcherbyna_2025}, but are not treated as the closest benchmark-protocol comparators in Table~\ref{tab:closest_work_delta}.
The concrete incremental delta claimed in this paper is integration-level: not a new planner, but a deterministic execution stack that couples seed propagation, heterogeneous output normalization, and geometric path validation in one auditable repository workflow.

\section{Method}
\subsection{Implemented System Architecture}
The implemented architecture is a modular runtime centered on \texttt{robotics\_maze/src/main.py}, with four concrete layers:
\begin{enumerate}
    \item \textbf{Orchestration layer} (\texttt{scripts/sim\_runner.py}, \texttt{main.py}): parses command-line arguments into \texttt{RunConfig}, supports optional GUI setup overrides, and executes an episode loop.
    \item \textbf{Maze and planning layer} (\texttt{maze.py}, \texttt{planners.py}, \texttt{alt\_planners/}): generates deterministic mazes and computes paths using baseline and alternative planners.
    \item \textbf{Simulation layer} (\texttt{sim.py}, \texttt{robot.py}, \texttt{geometry.py}): converts plans to waypoints, instantiates obstacles from maze walls, and executes motion in physics backends.
    \item \textbf{Evaluation layer} (\texttt{benchmark.py}): runs planner trials, validates returned paths, ranks methods, and writes CSV/Markdown artifacts.
\end{enumerate}

For episode \(e\in\{1,\dots,E\}\), \texttt{main.py} first computes an episode seed
\(s_e^{\mathrm{main}}=s_0+(e-1)\), where \(s_0\) is the user-provided base seed.
The current deterministic generator adapter then applies an additional episode offset,
yielding the effective maze-generation seed
\(s_e^{\mathrm{gen}}=s_e^{\mathrm{main}}+e=s_0+2e-1\).
This manuscript reports the behavior as currently implemented in code.

\subsection{Maze Generation and Grid Conversion}
The maze model (\texttt{Maze} dataclass) stores explicit horizontal and vertical wall arrays, start/goal cells, and generation metadata. Two algorithms are implemented: recursive backtracker and randomized Prim. After carving, generation is accepted only if (i) all \(W\times H\) cells are reachable from start and (ii) a BFS shortest path exists between start and goal. This provides a deterministic solvable-maze contract before planning is invoked.

For planner compatibility, \texttt{benchmark.py} converts wall-based mazes into occupancy grids of size \((2H+1)\times(2W+1)\), where free corridors are explicitly opened between neighboring cells. Start and goal are transformed from cell coordinates into occupancy-grid indices and forced free.

\subsection{Planner Interface and Normalization}
Baseline planners are registered by name in \texttt{planners.py} (A*, Dijkstra, BFS, and greedy best-first, plus aliases). Additional methods are loaded from \texttt{alt\_planners/} through module-symbol mapping. To support heterogeneous planner APIs, \texttt{FunctionPlannerAdapter} and benchmark-side normalization convert outputs to a common payload with path plus metrics (e.g., expansions and runtime).

Planner outputs are accepted from multiple schemas (dictionary, tuple, or raw path sequence), then normalized by \texttt{\_normalize\_planner\_output}. Path validity is enforced by \texttt{\_validate\_and\_measure\_path}: endpoints must match benchmark start/goal, all path cells must stay in bounds and collision free, and each segment is checked with Bresenham rasterization before success is credited.

\subsection{Simulation Backends and Controller Path}
\texttt{MazeEpisodeSimulator} receives \texttt{(maze, plan)} and performs three steps:
\begin{enumerate}
    \item extract or infer a path (\texttt{path}, \texttt{waypoints}, tuple payload, or maze fallback),
    \item map the path to world-frame waypoints (including occupancy-grid to metric conversion),
    \item execute via selected backend (\texttt{pybullet}, \texttt{mujoco}, or deterministic kinematic fallback).
\end{enumerate}

Backend selection is explicit: \texttt{auto} prefers PyBullet when available, then MuJoCo; forced backend requests degrade gracefully when dependencies are missing. For robot models, custom URDF loading is attempted first, then default fallback (\texttt{husky/husky.urdf}, then \texttt{r2d2.urdf}) to preserve run continuity. In PyBullet runs, a waypoint follower (\texttt{MobileRobotController}) applies heading-aware speed control, differential-drive wheel commands when available, and bounded command slew rates.

\subsection{Benchmark Protocol and Ranking}
Let \(\mathcal{P}\) denote the planner set and \(\mathcal{M}\) the generated mazes. The benchmark executes every \(p\in\mathcal{P}\) on every \(m\in\mathcal{M}\), rotating planner order per maze to reduce warm-start/caching bias. Each trial records success, solve time (ms), path length, expansions, and error text.
This structure aligns with recent benchmarking toolchains that separate dataset/task generation, planner execution normalization, and metric aggregation \cite{Heiden_2021,Chamzas_2022,Mayer_2024}.

To avoid unfair timing comparisons when planners fail on different mazes, the method computes a shared-success subset:
\[
\mathcal{M}_{\mathrm{shared}}=\{m\in\mathcal{M}\mid \forall p\in\mathcal{P},\; p \text{ succeeds on } m\}.
\]
Ranking follows the implemented lexicographic policy in \texttt{benchmark.py}:
\[
\text{rank}(p)\sim\big(-\mathrm{SR}_p,\; T^{\mathrm{shared}}_p,\; E_p,\; T_p,\; \mathrm{name}_p\big),
\]
where \(\mathrm{SR}\) is success rate, \(T^{\mathrm{shared}}\) comparable solve time, \(E\) mean expansions, and \(T\) mean solve time.
Planner name is used only as a deterministic final tie-break.
Comparable path length \(L^{\mathrm{shared}}\) is still reported as a descriptive metric, but excluded from ranking when any-angle planners are present to avoid mixed-geometry bias.

\subsubsection{Repeated-Run Timing Protocol}
To improve timing quality and enable rank-stability analysis, the benchmark supports a configurable repeated-run protocol via the \texttt{--repeats} and \texttt{--warmup} CLI arguments.
When \texttt{--repeats} \(R > 1\), each planner-maze pair is executed \(R\) timed times; an optional \(W \ge 0\) warm-up invocations (controlled by \texttt{--warmup}) are discarded before timing begins to mitigate cold-start and JIT effects.
Each timed repetition is recorded as a distinct row in \texttt{benchmark\_results.csv} (identified by the \texttt{repeat\_index} column), so all raw observations are retained for downstream analysis.

When \(R > 1\), two additional output artifacts are generated:
\begin{itemize}
    \item \textbf{\texttt{benchmark\_repeat\_stats.csv}}: per (planner, maze) aggregates across repeats---mean, median, standard deviation, minimum, and maximum of \texttt{solve\_time\_ms}.
    \item \textbf{\texttt{benchmark\_rank\_stability.md}}: pairwise Spearman rank-correlation coefficients \(\rho\) between every pair of repeat-round planner rankings, together with the mean \(\rho\) across all pairs as a single stability scalar.
\end{itemize}

A mean \(\rho\) near \(1.0\) indicates that the rank ordering of planners is consistent across timed repetitions; values substantially below \(1.0\) suggest that the timing spread is wide enough to make per-repeat rankings unreliable.

This protocol outputs \texttt{benchmark\_results.csv} and \texttt{benchmark\_summary.md}, enabling traceable comparison and direct linkage between manuscript claims and generated artifacts.

\section{Experimental Setup}
\label{sec:experiments}

This section documents (i) experiments already executed in the current repository snapshot and
(ii) planned studies that are intentionally out of scope for the current results.

\subsection{Executed Experiments}
\textbf{System and workflow context (executed).}
Figure~\ref{fig:system_pipeline} summarizes the deterministic runtime and artifact path used for
the reported experiments, from configuration parsing and planner execution through benchmark export.
Manuscript-process coordination artifacts are tracked in \texttt{coordination/} files and are not
treated as primary scientific evidence in the main experimental narrative.

\begin{figure*}[t]
\centering
\includegraphics[width=\textwidth]{figures/system_pipeline.png}
\caption{Deterministic system pipeline used in the executed experiments, including planner
benchmarking and artifact emission. The runtime path captures backend fallback behavior
(\texttt{pybullet} \(\rightarrow\) \texttt{mujoco} \(\rightarrow\) deterministic fallback) and
the benchmark output branch used for reproducible reporting.}
\label{fig:system_pipeline}
\end{figure*}

\textbf{Planner benchmark (executed).}
We evaluate planners with the repository benchmark harness
(\texttt{robotics\_maze/src/benchmark.py}). The harness generates mazes with fixed settings
(\(50\) mazes, \(15 \times 15\) cells, \texttt{backtracker}, base seed \(7\), per-maze seed
\(= 7 + \texttt{maze\_index}\)) and, by default, fail-closed enforces the canonical 12-planner
evaluation set used in this paper. In the committed benchmark
artifacts (\texttt{robotics\_maze/results/benchmark\_summary.md} and
\texttt{robotics\_maze/results/benchmark\_results.csv}), this corresponds to 12 planners:
\texttt{astar}, \texttt{dijkstra}, \texttt{greedy\_best\_first},
\texttt{r1\_weighted\_astar}, \texttt{r2\_bidirectional\_astar}, \texttt{r3\_theta\_star},
\texttt{r4\_idastar}, \texttt{r5\_jump\_point\_search}, \texttt{r6\_lpa\_star},
\texttt{r7\_beam\_search}, \texttt{r8\_fringe\_search}, and
\texttt{r9\_bidirectional\_bfs}.
The baseline BFS implementation exists in the planner registry but is intentionally excluded from
the benchmark harness default discovered set; all reported rankings and tables use this 12-planner
execution set for consistency with committed artifacts.

Per trial, the harness records success, wall-clock solve time (ms), path length, and node
expansions. Reported paths are validated against occupancy and bounds constraints, including
segment-level collision checks; a trial is marked successful only if both planner output and path
validation succeed. Solve time is measured using Python \texttt{time.perf\_counter()} around each
planner invocation and converted to milliseconds. To reduce first-run cache bias, planner execution
order is rotated per maze. In the current snapshot, each planner-maze pair is executed once, so
timing differences should be interpreted as descriptive. Aggregated ranking follows the executable
lexicographic order in \texttt{benchmark.py}: success rate, comparable shared-success solve time,
mean expansions, overall mean solve time, then planner name as deterministic tie-break.
Comparable path length is reported but excluded from ranking because any-angle and cardinal-grid
path geometries are not directly commensurate.

\textbf{Deterministic simulation regression (executed).}
In addition to planner benchmarking, the repository executes deterministic simulation checks
(\texttt{robotics\_maze/testing/run\_sim\_tests.sh}) for three representative planners
(\texttt{astar}, \texttt{weighted\_astar}, \texttt{fringe\_search}) over three episodes
(\texttt{maze-size}=11, \texttt{seed}=42). The same test run invokes the deterministic screenshot
pipeline (\texttt{robotics\_maze/scripts/capture\_regression\_screenshots.py}), which enforces six
expected PNG outputs (three MuJoCo and three fallback renderer images). The latest committed test
log reports all deterministic runs and screenshot checks as PASS.

\textbf{Dynamic-environment benchmark extension (executed).}
We extend the benchmark to include controlled dynamic disturbances where obstacles are introduced
along the planner's initial optimal path. The benchmark harness simulates robot progress and
triggers replanning upon obstacle detection. We record new performance metrics: replanning latency,
mean replans per episode, collision count, and progress-to-goal. This extension allows quantifying
planner robustness and recovery behavior under online disturbances, providing a basis for
evaluating incremental replanning efficiency.

\begin{table*}[t]
\centering
\caption{Reproducibility protocol, explicitly separated into executed vs planned studies.}
\label{tab:experiment_protocol}
\begin{tabular}{p{0.19\textwidth} p{0.38\textwidth} p{0.38\textwidth}}
\toprule
Component & Executed protocol (current artifacts) & Planned protocol (not yet executed) \\
\midrule
Entrypoints &
\texttt{python robotics\_maze/src/benchmark.py} for planner benchmarks; \texttt{bash robotics\_maze/testing/run\_sim\_tests.sh} for deterministic simulation and screenshot regression. &
Add CI jobs that run full benchmark and regression protocols on pinned hardware classes and publish signed artifacts per run. \\
\midrule
Benchmark workload &
50 mazes, \(15 \times 15\), \texttt{backtracker}, base seed 7, maze seed = \(7 +\) maze index. Current snapshot evaluates 12 planners. &
Scale to larger mazes and multi-generator sweeps (\texttt{backtracker} + \texttt{prim}) with expanded seed ranges to evaluate ranking stability. \\
\midrule
Simulation regression workload &
3 planners (\texttt{astar}, \texttt{weighted\_astar}, \texttt{fringe\_search}), 3 episodes each, \texttt{maze-size}=11, \texttt{seed}=42, backend preference \texttt{auto}. &
Matched paired runs with forced \texttt{pybullet} and \texttt{mujoco}; additional stress scenarios with dynamic obstacles and replanning triggers. \\
\midrule
Primary metrics &
Per trial: success, solve\_time\_ms, path\_length, expansions, and error text on failure. Summary ranking prioritizes success, then comparable shared-success solve time, mean expansions, and overall mean solve time (planner name as deterministic tie-break). Path length is reported descriptively. &
Add variance statistics (median, p95), per-backend deltas, and disturbance-recovery metrics for dynamic scenes. \\
\midrule
Validity checks &
Path-level validation enforces endpoint consistency, in-bounds traversal, and collision-free segments; benchmark marks success only when planner and validator agree. &
Add robustness checks for near-collision margins, execution drift under dynamics, and failure-mode taxonomy by planner family. \\
\midrule
Visual regression &
Deterministic screenshot pipeline requires six fixed filenames (3 MuJoCo + 3 fallback). Latest committed run reports PASS on all checks. &
Add pixel-diff thresholds and automated anomaly triage reports in CI for each benchmark commit. \\
\bottomrule
\end{tabular}
\end{table*}

Table~\ref{tab:experiment_protocol} separates executed protocol components (used for all current
claims) from planned, not-yet-executed studies.

\subsection{Planned Future Experiments (Not Yet Executed)}
The following experiments are planned and are \textbf{not} included in the current result tables:
\begin{enumerate}
    \item \textbf{Cross-algorithm robustness sweep:} expand benchmark runs across
    \texttt{backtracker} and \texttt{prim} generators, larger mazes, and broader seed sets to
    quantify ranking stability.
    \item \textbf{Post-2021 planner integration study:} add learned local heuristics, Hybrid-A*,
    and uncertainty-aware planning baselines from the repository shortlist, then rerun the same
    benchmark protocol for direct comparability.
    \item \textbf{Backend sensitivity analysis:} run matched scenarios with explicit
    \texttt{pybullet} and \texttt{mujoco} settings to isolate simulator-dependent effects on
    runtime and path execution behavior.
\end{enumerate}
No quantitative claims from these planned experiments are used in the present manuscript version.

\section{Results}
\label{sec:results}

Table~\ref{tab:main_results} summarizes planner performance on 50 generated \(15\times15\) backtracker mazes (seeds 7--56). All 12 planners solved all trials (600/600) with no runtime errors, so differences here primarily reflect computational efficiency rather than reachability.

\begin{table*}[t]
\centering
\caption{Main benchmark results on 50 generated \(15\times15\) backtracker mazes. Rows are ranked by success rate (descending), then comparable mean solve time on shared-success mazes, mean expansions, and overall mean solve time (ascending), with planner name as deterministic tie-break. Time is reported as mean \(\pm\) standard deviation over mazes; median and interquartile range (IQR) are included to expose skew. Lower is better for time, path length, and expansions.}
\label{tab:main_results}
\footnotesize
\setlength{\tabcolsep}{4.2pt}
\begin{tabular}{clccccc}
\toprule
Rank & Planner & Success & Time (ms) $\downarrow$ & Median [IQR] (ms) $\downarrow$ & Path Length $\downarrow$ & Expansions $\downarrow$ \\
\midrule
1  & R1 Weighted A*         & 50/50 & 0.35 $\pm$ 0.21   & 0.29 [0.15, 0.54]   & 142.72          & 187.16  \\
2  & R7 Beam Search         & 50/50 & 0.42 $\pm$ 0.24   & 0.38 [0.20, 0.65]   & 142.72          & 198.36  \\
3  & R5 Jump Point Search   & 50/50 & 0.45 $\pm$ 0.26   & 0.39 [0.19, 0.69]   & 142.72          & 57.26   \\
4  & Greedy Best-First      & 50/50 & 0.46 $\pm$ 0.27   & 0.38 [0.21, 0.71]   & 142.72          & 171.96  \\
5  & R8 Fringe Search       & 50/50 & 0.52 $\pm$ 0.31   & 0.43 [0.23, 0.79]   & 142.72          & 190.30  \\
6  & A*                     & 50/50 & 0.52 $\pm$ 0.31   & 0.43 [0.22, 0.79]   & 142.72          & 189.06  \\
7  & Dijkstra               & 50/50 & 0.54 $\pm$ 0.32   & 0.48 [0.22, 0.84]   & 142.72          & 200.52  \\
8  & R9 Bidirectional BFS   & 50/50 & 0.54 $\pm$ 0.31   & 0.49 [0.24, 0.80]   & 142.72          & 201.52  \\
9  & R2 Bidirectional A*    & 50/50 & 1.25 $\pm$ 0.69   & 1.28 [0.52, 1.92]   & 142.72          & 468.02  \\
10 & R3 Theta*              & 50/50 & 1.55 $\pm$ 0.92   & 1.27 [0.65, 2.35]   & 97.96$^\dagger$ & 189.18  \\
11 & R6 LPA*                & 50/50 & 3.95 $\pm$ 1.59   & 3.56 [2.41, 5.59]   & 142.72          & 295.10  \\
12 & R4 IDA*                & 50/50 & 22.56 $\pm$ 22.13 & 11.05 [2.93, 43.00] & 142.72          & 7061.34 \\
\bottomrule
\end{tabular}

{\raggedright\footnotesize $^\dagger$Theta* uses any-angle motion, so path-length values are not directly comparable to cardinal-grid planners.\par}
\end{table*}

Figures~\ref{fig:benchmark_runtime_ms}, \ref{fig:runtime_uncertainty}, \ref{fig:benchmark_expansions}, and \ref{fig:benchmark_success_rate} provide complementary visual summaries of runtime, runtime uncertainty, search effort, and success rate using the same benchmark snapshot as Table~\ref{tab:main_results}.

\begin{figure}[t]
\centering
\includegraphics[width=\columnwidth]{figures/benchmark_runtime_ms.png}
\caption{Mean planner solve time (ms) on a logarithmic scale over 50 benchmark mazes. Lower values
indicate faster planning.}
\label{fig:benchmark_runtime_ms}
\end{figure}

\begin{figure}[t]
\centering
\includegraphics[width=\columnwidth]{figures/runtime_uncertainty.png}
\caption{Runtime uncertainty over the same 50 paired mazes. Horizontal box summaries are shown on a logarithmic scale (box: IQR, center line: median, red dot: mean).}
\label{fig:runtime_uncertainty}
\end{figure}

\begin{figure}[t]
\centering
\includegraphics[width=\columnwidth]{figures/benchmark_expansions.png}
\caption{Mean node expansions on a logarithmic scale for the same benchmark runs. Lower values
indicate lower search effort.}
\label{fig:benchmark_expansions}
\end{figure}

\begin{figure}[t]
\centering
\includegraphics[width=\columnwidth]{figures/benchmark_success_rate.png}
\caption{Planner success rate over 50 mazes. All methods achieve \(100\%\) in this static benchmark
setting.}
\label{fig:benchmark_success_rate}
\end{figure}

\paragraph{Overall ranking and runtime spread.}
\texttt{r1\_weighted\_astar} is fastest in mean solve time (0.35 ms), followed by \texttt{r7\_beam\_search} (0.42 ms). A second tier---\texttt{r5\_jump\_point\_search}, \texttt{greedy\_best\_first}, \texttt{r8\_fringe\_search}, \texttt{astar}, \texttt{dijkstra}, and \texttt{r9\_bidirectional\_bfs}---is tightly clustered between 0.45 and 0.54 ms (at most +0.19 ms versus the top row). At maze level, \texttt{r1\_weighted\_astar} is fastest on 48/50 mazes, but the best-vs-second-best margin is small (median 0.049 ms; 43/50 mazes within 0.1 ms), indicating limited practical separation among the fastest methods in this setup.
This narrow spread is visible in Figures~\ref{fig:benchmark_runtime_ms} and~\ref{fig:runtime_uncertainty}.

\paragraph{Inferential runtime comparison.}
To characterize runtime consistency across paired mazes, each planner was compared against \texttt{r1\_weighted\_astar} using exact two-sided paired sign tests with Holm correction (family-wise \(\alpha=0.05\)). Effect sizes are reported as paired median runtime deltas (\(\Delta=\text{comparator}-\texttt{r1\_weighted\_astar}\), ms) with 95\% bootstrap confidence intervals from 40{,}000 paired resamples.
\begin{table*}[t]
\centering
\caption{Exploratory paired runtime comparisons against \texttt{r1\_weighted\_astar} on the same 50 mazes (single run per planner-maze pair). Positive \(\Delta\) means the comparator is slower. Confidence intervals are percentile bootstrap intervals from 40{,}000 paired resamples (fixed seed). \(p\)-values are exact two-sided paired sign tests with Holm correction across 11 comparisons.}
\label{tab:runtime_statistical_comparison}
\footnotesize
\setlength{\tabcolsep}{4.0pt}
\begin{tabular}{lcccc}
\toprule
Comparator & Median \(\Delta\) (ms) & 95\% CI for \(\Delta\) (ms) & Slower/Faster (of 50) & Holm-adjusted \(p\) \\
\midrule
A*                       & 0.173  & [0.121, 0.248]         & 50/0 & \(1.95\times10^{-14}\) \\
Dijkstra                 & 0.162  & [0.129, 0.255]         & 50/0 & \(1.95\times10^{-14}\) \\
Greedy Best-First        & 0.094  & [0.064, 0.114]         & 47/3 & \(3.71\times10^{-11}\) \\
R2 Bidirectional A*      & 1.183  & [0.686, 1.540]         & 50/0 & \(1.95\times10^{-14}\) \\
R3 Theta*                & 1.078  & [0.784, 1.747]         & 50/0 & \(1.95\times10^{-14}\) \\
R4 IDA*                  & 14.777 & [6.989, 37.396]        & 50/0 & \(1.95\times10^{-14}\) \\
R5 Jump Point Search     & 0.065  & [0.051, 0.103]         & 50/0 & \(1.95\times10^{-14}\) \\
R6 LPA*                  & 3.609  & [3.037, 4.703]         & 50/0 & \(1.95\times10^{-14}\) \\
R7 Beam Search           & 0.049  & [0.037, 0.070]         & 50/0 & \(1.95\times10^{-14}\) \\
R8 Fringe Search         & 0.174  & [0.122, 0.255]         & 50/0 & \(1.95\times10^{-14}\) \\
R9 Bidirectional BFS     & 0.189  & [0.128, 0.217]         & 50/0 & \(1.95\times10^{-14}\) \\
\bottomrule
\end{tabular}
\end{table*}

For the closest comparator (\texttt{r7\_beam\_search}), the median paired delta is 0.068 ms (95\% CI [0.046, 0.085]) with \texttt{r1\_weighted\_astar} faster on 50/50 mazes; this indicates a consistent but small absolute gain in this dataset. Larger separations appear for slower planners (e.g., \texttt{r6\_lpa\_star}: 3.30 ms [2.73, 4.27]; \texttt{r4\_idastar}: 10.76 ms [5.70, 30.03]). Because each planner-maze pair is measured once, these inferential statistics should be interpreted as exploratory consistency indicators, not hardware-controlled latency certification.

\paragraph{Search-effort trade-offs.}
\texttt{r5\_jump\_point\_search} attains the lowest expansion count (57.26) while remaining in the fast runtime cluster (0.45 ms), reducing expansions by roughly 69.7\% relative to baseline \texttt{astar} (189.06). \texttt{r2\_bidirectional\_astar} and \texttt{r3\_theta\_star} are slower (1.25 and 1.55 ms). \texttt{r6\_lpa\_star} and \texttt{r4\_idastar} are clear outliers at 3.95 ms and 22.56 ms, with \texttt{r4\_idastar} also showing the highest variability (std 22.13 ms; IQR 2.93--43.00 ms) and expansion count (7061.34).
The expansion profile in Figure~\ref{fig:benchmark_expansions} highlights this separation between
the efficient frontier (\texttt{r5\_jump\_point\_search}) and high-overhead outliers.

\paragraph{Uncertainty and limitations.}
These results are limited to one benchmark regime: 50 mazes, one maze size (\(15\times15\)), one generator (backtracker), and one seed schedule. Timings are wall-clock and mostly sub-millisecond for the fastest methods, so statistical significance should not be conflated with large practical effect size in deployment settings. The benchmark also covers static, fully known mazes only; conclusions may not transfer to dynamic, partially observable, or larger-scale environments. Finally, Theta* uses any-angle motion, so its path-length values are not directly comparable to cardinal-grid planners.
The flat success-rate profile in Figure~\ref{fig:benchmark_success_rate} further emphasizes that
this dataset mainly distinguishes planners by efficiency rather than solvability.

\section{Discussion}

The current repository snapshot establishes a reproducible baseline for grid-maze navigation with measurable algorithmic tradeoffs. The benchmark artifact reports 12 planners evaluated on 50 generated $15\times15$ mazes, with 100\% success for every planner in this static setting. Under the executable ranking policy (success rate, then comparable solve time on shared-success mazes, then mean expansions, then overall mean solve time, then planner-name tie-break), \texttt{r1\_weighted\_astar} is first (0.35 ms mean solve time), while \texttt{r4\_idastar} is last (22.56 ms), showing that admissible but deep iterative search remains expensive for this map class.

\subsection{What the Current Results Establish}

Three descriptive findings are consistent within the present benchmark regime.
First, weighted heuristic guidance is already beneficial: \texttt{r1\_weighted\_astar} is approximately 35\% faster than baseline \texttt{astar} while keeping equivalent mean path length in this dataset.
Second, frontier-pruning methods reduce search effort but do not always dominate runtime: \texttt{r5\_jump\_point\_search} has the lowest mean expansions (57.26) yet is not the fastest due to additional bookkeeping overhead.
Third, algorithm classes optimized for other settings underperform in this static benchmark; for example, \texttt{r6\_lpa\_star} and \texttt{r4\_idastar} add overhead without gains when maps are regenerated rather than incrementally repaired.
These observations are descriptive summaries from the executed benchmark snapshot and are not
presented as inferential superiority claims.

Beyond planner speed, the infrastructure contributes practical value.
The simulator supports explicit backend selection (\texttt{pybullet}, \texttt{mujoco}, or \texttt{auto}), URDF fallback behavior, and deterministic screenshot regression capture for visual checks.
These implementation details improve reproducibility and reduce benchmark fragility across machines.

\subsection{Implemented Versus Not-Yet-Implemented Methods}

\textbf{Implemented and used in the reported benchmark.}
\begin{itemize}
    \item Baseline grid planners: \texttt{astar}, \texttt{dijkstra}, \texttt{greedy\_best\_first}.
    \item Alternative planners: \texttt{r1\_weighted\_astar}, \texttt{r2\_bidirectional\_astar}, \texttt{r3\_theta\_star}, \texttt{r4\_idastar}, \texttt{r5\_jump\_point\_search}, \texttt{r6\_lpa\_star}, \texttt{r7\_beam\_search}, \texttt{r8\_fringe\_search}, and \texttt{r9\_bidirectional\_bfs}.
    \item Runtime stack support: backend auto-resolution, robust URDF fallback, and deterministic screenshot regression scripts.
\end{itemize}

\textbf{Implemented in repository but not part of the current summary table.}
\begin{itemize}
    \item Additional adapter modules \texttt{r11\_dijkstra}, \texttt{r12\_bfs}, and \texttt{r13\_greedy\_best\_first} exist, but the current benchmark artifact reports the baseline/alternative set above.
\end{itemize}

\textbf{Not yet implemented (currently documented as future directions).}
\begin{itemize}
    \item Learned local heuristics for search (LoHA-style A* extensions).
    \item D* Lite + DWA and AD* + DWA hybrid global-local replanning.
    \item Heading-aware SE(2) Hybrid-A* / state-lattice planning and IGHA* style incremental hybrid search.
    \item Guided RRT* variants and uncertainty-aware MPC for dynamic, continuous navigation.
\end{itemize}

\subsection{Limitations and Threats to Validity}

The present evidence is strong for reproducible static-grid comparison, but limited for broader robotics claims.
\begin{itemize}
    \item \textbf{Task distribution}: all reported mazes are $15\times15$ and generated by one algorithm family in one snapshot, so topology diversity is limited.
    \item \textbf{Metric saturation}: with 100\% success across planners, failure robustness cannot be ranked; only efficiency metrics differentiate methods.
    \item \textbf{Path comparability}: any-angle methods (e.g., \texttt{r3\_theta\_star}) are not directly comparable to strict lattice paths without additional smoothness/feasibility metrics.
    \item \textbf{Physics realism}: MuJoCo fallback currently executes a simplified waypoint progression, and no dynamic-obstacle uncertainty benchmark is yet integrated.
    \item \textbf{Generalization}: current conclusions are for occupancy-grid planning and do not yet establish superiority in SE(2)/kinodynamic domains.
\end{itemize}

\subsection{Future Work}

The next implementation priorities are: (i) learned local heuristics with admissibility safeguards, (ii) incremental global-local replanning under dynamic disturbances, (iii) heading-aware SE(2) planners for executable trajectory quality, and (iv) uncertainty-aware continuous planners for large-map robustness. All four directions require the same acceptance discipline before being promoted into default comparisons: multi-generator and multi-seed evaluation, repeated-run timing protocols, and feasibility-oriented metrics beyond grid-path length.

These items are intentionally framed as planned work rather than current contributions. The present manuscript claims are limited to the executed static-grid benchmark infrastructure and its repository-grounded evidence.

\section{Conclusion}

This manuscript stage delivers a reproducible baseline for maze-navigation planning with explicit evidence from code and benchmark artifacts. In the current snapshot, 12 implemented planners solve all 50 static $15\times15$ benchmark mazes, with \texttt{r1\_weighted\_astar} leading runtime and \texttt{r4\_idastar} showing the highest computational burden. The implemented system is strongest today as a controlled occupancy-grid comparison and simulation harness (backend fallback, URDF robustness, and deterministic visual-regression support), not yet as a fully dynamic or kinodynamically realistic planner stack.

Method scope is therefore explicit: classical and alternative grid-search planners are implemented and benchmarked, while LoHA-style learned heuristics, D* Lite/AD* + DWA coupling, SE(2) Hybrid-A* or IGHA*, guided RRT*, and uncertainty-aware MPC remain not-yet-implemented research directions. The next milestone is to convert these forward hypotheses into acceptance-tested experiments with dynamic obstacles, larger maps, and feasibility-oriented metrics so future claims are supported by statistically grounded evidence rather than extrapolation.

\appendices
\section{Additional Details}
\subsection{Notation and Occupancy-Grid Construction}
Let the maze have width $W$ and height $H$ in cell coordinates. The benchmark converts each maze to a binary occupancy lattice
\begin{equation}
G \in \{0,1\}^{R \times C}, \quad R = 2H + 1, \quad C = 2W + 1,
\end{equation}
where $G_{r,c}=0$ denotes free space and $G_{r,c}=1$ denotes blocked space.
The cell-to-lattice map used by the implementation is
\begin{equation}
\phi(x,y) = (2y+1,\,2x+1),
\end{equation}
so the benchmark start and goal nodes are
\begin{equation}
s=\phi(x_s,y_s), \qquad g=\phi(x_g,y_g).
\end{equation}

For a lattice node $n=(r,c)$, the neighborhood is either 4-connected or 8-connected:
\begin{equation}
\mathcal{N}_4(n)=\{(r-1,c),(r,c+1),(r+1,c),(r,c-1)\}\cap\Omega,
\end{equation}
\begin{equation}
\mathcal{N}_8(n)=\mathcal{N}_4(n)\cup\{(r-1,c-1),(r-1,c+1),(r+1,c+1),(r+1,c-1)\}\cap\Omega,
\end{equation}
with $\Omega$ the in-bounds lattice set.
The step cost in the baseline planners is
\begin{equation}
c(u,v)=
\begin{cases}
\sqrt{2}, & \text{if }u\rightarrow v\text{ is diagonal},\\
1, & \text{otherwise}.
\end{cases}
\end{equation}

Heuristic options are
\begin{equation}
h_{\mathrm{man}}(n,g)=|r-r_g|+|c-c_g|,
\end{equation}
\begin{equation}
h_{\mathrm{euc}}(n,g)=\sqrt{(r-r_g)^2+(c-c_g)^2},
\end{equation}
\begin{equation}
h_{\mathrm{cheb}}(n,g)=\max\{|r-r_g|,|c-c_g|\}.
\end{equation}

\subsection{Planner and Benchmark Objectives}
For a path $\pi=(n_0,\dots,n_K)$ with $n_0=s$ and $n_K=g$, the weighted path-cost objective is
\begin{equation}
J_{\mathrm{cost}}(\pi)=\sum_{k=1}^{K} c(n_{k-1},n_k).
\end{equation}

The baseline priority keys used in the benchmark are:
\begin{equation}
f_{\mathrm{A*}}(n)=g(n)+w\,h(n), \qquad w\ge 1,
\end{equation}
\begin{equation}
f_{\mathrm{Dijkstra}}(n)=g(n),
\end{equation}
\begin{equation}
f_{\mathrm{GBFS}}(n)=h(n),
\end{equation}
and BFS minimizes hop count
\begin{equation}
J_{\mathrm{hop}}(\pi)=K.
\end{equation}

For A*, the secondary tie key is
\begin{equation}
\tau(n)=
\begin{cases}
h(n), & \text{low\_h},\\
-g(n), & \text{high\_g},\\
0, & \text{fifo},
\end{cases}
\end{equation}
and the heap key is $(f(n),\tau(n),t(n))$ where $t(n)$ is a monotonically increasing insertion counter.

\paragraph*{Path validation and measured length}
Given returned path $P=(p_0,\dots,p_K)$, validity requires:
\begin{equation}
p_0=s,\quad p_K=g,\quad p_k\in\Omega,\quad G_{p_k}=0,\ \forall k,
\end{equation}
and for each segment, every rasterized Bresenham cell is also in-bounds and free.
If $\mathcal{B}(p_{k-1},p_k)$ denotes the rasterized segment cell sequence, the measured benchmark path length is
\begin{equation}
L(P)=\sum_{k=1}^{K}\left(\left|\mathcal{B}(p_{k-1},p_k)\right|-1\right).
\end{equation}

\paragraph*{Planner ranking objective}
For planner $p$ over $M$ mazes, with success indicator $I_{p,m}\in\{0,1\}$, solve time $t_{p,m}$, measured length $L_{p,m}$, and expansions $e_{p,m}$:
\begin{equation}
S_p=\frac{1}{M}\sum_{m=1}^{M} I_{p,m},\qquad
\mathcal{M}_p^+=\{m\mid I_{p,m}=1\}.
\end{equation}
Define the shared-success set
\begin{equation}
\mathcal{M}_{\mathrm{shared}}=\{m\mid I_{q,m}=1,\ \forall q\in\mathcal{P}\}.
\end{equation}
The comparable means used for ranking are
\begin{equation}
\bar{t}_p^{c}=
\begin{cases}
\frac{1}{|\mathcal{M}_{\mathrm{shared}}|}\sum_{m\in\mathcal{M}_{\mathrm{shared}}} t_{p,m}, & |\mathcal{M}_{\mathrm{shared}}|>0,\\
\frac{1}{M}\sum_{m=1}^{M} t_{p,m}, & \text{otherwise},
\end{cases}
\end{equation}
\begin{equation}
\bar{L}_p^{c}=
\begin{cases}
\frac{1}{|\mathcal{M}_{\mathrm{shared}}|}\sum_{m\in\mathcal{M}_{\mathrm{shared}}} L_{p,m}, & |\mathcal{M}_{\mathrm{shared}}|>0,\\
\frac{1}{|\mathcal{M}_p^{+}|}\sum_{m\in\mathcal{M}_p^{+}} L_{p,m}, & \text{otherwise}.
\end{cases}
\end{equation}
With
\begin{equation}
\bar{e}_p=\frac{1}{|\mathcal{M}_p^{+}|}\sum_{m\in\mathcal{M}_p^{+}} e_{p,m}, \qquad
\bar{t}_p=\frac{1}{M}\sum_{m=1}^{M} t_{p,m},
\end{equation}
ranking is lexicographic on
\begin{equation}
\left(-S_p,\ \bar{t}_p^{c},\ \bar{L}_p^{c},\ \bar{e}_p,\ \bar{t}_p,\ \text{name}_p\right).
\end{equation}

\subsection{Local Control Abstraction}
\paragraph*{Path-to-waypoint map}
The simulator converts planner outputs to world-frame waypoints.
If a path is recognized as cell-grid coordinates $(x_k,y_k)$:
\begin{equation}
w_k=\left((x_k+0.5)s_c,\ (y_k+0.5)s_c\right),
\end{equation}
where $s_c$ is \texttt{cell\_size}.
If recognized as occupancy-lattice coordinates:
\begin{equation}
w_k=\left(0.5\,s_c\,x_k,\ 0.5\,s_c\,y_k\right).
\end{equation}
Cell-grid waypoints are additionally compressed by removing collinear interior points.

\paragraph*{Waypoint tracking law}
At control step $t$, with robot pose $(x_t,y_t,\psi_t)$ and active waypoint $(x_j,y_j)$:
\begin{equation}
d_t=\sqrt{(x_j-x_t)^2+(y_j-y_t)^2},
\end{equation}
\begin{equation}
\theta_t=\operatorname{atan2}(y_j-y_t,\ x_j-x_t),\qquad
e_{\psi,t}=\mathrm{wrap}_{[-\pi,\pi]}(\theta_t-\psi_t).
\end{equation}

The angular command is the clipped proportional target
\begin{equation}
\omega_t^\star=\arg\min_{|\omega|\le \omega_{\max}}|\omega-k_\psi e_{\psi,t}|
=\mathrm{clip}(k_\psi e_{\psi,t},-\omega_{\max},\omega_{\max}).
\end{equation}

The linear target is the largest feasible speed under speed, distance, and braking bounds:
\begin{equation}
v_t^\star=\alpha_t\cdot \max_{v\ge 0} v
\quad\text{s.t.}\quad
v\le v_{\max},\ \ v\le d_t,\ \ v\le\sqrt{2a_{\mathrm{dec}}d_t},
\end{equation}
equivalently
\begin{equation}
v_t^\star=\alpha_t\cdot\min\left\{v_{\max},\ d_t,\ \sqrt{2a_{\mathrm{dec}}d_t}\right\}.
\end{equation}
The heading gate $\alpha_t\in[0,1]$ is
\begin{equation}
\alpha_t=
\begin{cases}
0, & |e_{\psi,t}|\ge \theta_{\mathrm{turn}},\\
1, & |e_{\psi,t}|\le \theta_{\mathrm{drive}},\\
1-\sigma\!\left(\dfrac{|e_{\psi,t}|-\theta_{\mathrm{drive}}}{\theta_{\mathrm{turn}}-\theta_{\mathrm{drive}}}\right), & \text{otherwise},
\end{cases}
\end{equation}
with smoothstep
\begin{equation}
\sigma(z)=z^2(3-2z), \quad z\in[0,1].
\end{equation}

Commands are passed through acceleration/deceleration slew limits with control step $\Delta t$:
\begin{equation}
u_t=u_{t-1}+\mathrm{clip}\!\left(u_t^\star-u_{t-1},-\Delta_u,\Delta_u\right),
\end{equation}
where
\begin{equation}
\Delta_u=
\begin{cases}
a_{u,+}\Delta t, & \operatorname{sgn}(u_t^\star)=\operatorname{sgn}(u_{t-1})\ \wedge\ |u_t^\star|>|u_{t-1}|,\\
a_{u,-}\Delta t, & \text{otherwise},
\end{cases}
\end{equation}
applied separately to $u=v$ and $u=\omega$.

Then the turn-rate-dependent linear cap is enforced:
\begin{equation}
\eta_t=\mathrm{clip}\!\left(\frac{|\omega_t|}{\omega_{\max}},0,1\right),\qquad
v_{\mathrm{turn},t}=v_{\max}\max\left(0.18,\ 1-0.68\,\sigma(\eta_t)\right),
\end{equation}
\begin{equation}
v_t\leftarrow \mathrm{clip}(v_t,-v_{\mathrm{turn},t},v_{\mathrm{turn},t}).
\end{equation}

For differential drive with axle track $b$ and wheel radius $r$:
\begin{equation}
v_{L,t}=v_t-\frac{b}{2}\omega_t,\qquad
v_{R,t}=v_t+\frac{b}{2}\omega_t,
\end{equation}
\begin{equation}
\Omega_{L,t}=v_{L,t}/r,\qquad \Omega_{R,t}=v_{R,t}/r.
\end{equation}
Waypoint index $j$ is advanced when $d_t\le \varepsilon_j$, where $\varepsilon_j$ is the per-waypoint tolerance (larger at the final waypoint).

\subsection{Complexity Statements}
Let $V=RC$ and $E\le 8V$ for the occupancy lattice graph.

\begin{equation}
T_{\mathrm{A*}/\mathrm{Dijkstra}/\mathrm{GBFS}}=O(E\log V),\qquad
M_{\mathrm{A*}/\mathrm{Dijkstra}/\mathrm{GBFS}}=O(V),
\end{equation}
because open lists are binary heaps and closed/g-score/came-from are hash maps/sets.

\begin{equation}
T_{\mathrm{BFS}}=O(V+E),\qquad M_{\mathrm{BFS}}=O(V),
\end{equation}
from deque frontier plus visited/parent maps.

For one returned path $P$:
\begin{equation}
T_{\mathrm{validate}}=O\!\left(\sum_{k=1}^{K}\left|\mathcal{B}(p_{k-1},p_k)\right|\right)
=O(L(P)).
\end{equation}

For $M$ mazes and $|\mathcal{P}|$ planners:
\begin{equation}
T_{\mathrm{benchmark}}=
O\!\left(
\sum_{m=1}^{M}\sum_{p\in\mathcal{P}}
\left(T_{\mathrm{plan}}(p,m)+T_{\mathrm{validate}}(p,m)\right)
\right).
\end{equation}

Per control update, command synthesis is constant-time, and actuator dispatch is linear in wheel-joint count $n_w$:
\begin{equation}
T_{\mathrm{control\ step}}=O(1+n_w).
\end{equation}

\subsection{Reproducibility Checklist}
\label{app:reproducibility_checklist}

\begin{table}[t]
\centering
\caption{Repository-grounded reproducibility checklist snapshot (2026-02-26).}
\begin{tabular}{p{0.28\linewidth}p{0.12\linewidth}p{0.54\linewidth}}
\toprule
Item & Status & Evidence \\
\midrule
Environment specification and dependencies & complete & \texttt{pixi.toml} (workspace tasks), \texttt{robotics\_maze/pixi.toml} (Python 3.11, PyBullet, MuJoCo, pytest). \\
CLI entrypoints and run commands & complete & \texttt{scripts/sim\_runner.py} and root \texttt{pixi.toml} task definitions. \\
Physics backend fallback behavior & complete & \texttt{robotics\_maze/src/sim.py} backend resolution and deterministic fallback branch. \\
Planner benchmark protocol and ranking policy & complete & \texttt{robotics\_maze/src/benchmark.py}; generated \texttt{robotics\_maze/results/benchmark\_summary.md}. \\
Benchmark trial artifacts & complete & \texttt{robotics\_maze/results/benchmark\_results.csv} and \texttt{robotics\_maze/results/benchmark\_summary.md}. \\
Regression screenshot generation & complete & \texttt{robotics\_maze/scripts/capture\_regression\_screenshots.py} strict expected filename checks. \\
Smoke and regression test logs & complete & \texttt{robotics\_maze/tests/test\_core.py} and \texttt{robotics\_maze/testing/TEST\_RUN\_LOG.md}. \\
Reference inventory target (\(\geq\)40 refs from 2021+) & complete & \texttt{references.bib} and \texttt{coordination/citations\_audit.csv} contain populated, post-2021 citation records in the current snapshot. \\
Citation quality target (\(\geq\)80\% peer reviewed) & complete & \texttt{coordination/citations\_audit.csv} marks peer-review status per entry and is populated beyond header-only state. \\
Figure manifest inventory completeness & complete & \texttt{coordination/figure\_manifest.csv} lists tracked figure IDs, source paths, and reproducibility fields. \\
Claim-traceability freshness gate & complete & \texttt{coordination/claims\_traceability.csv} refreshed to verified entries for current manuscript claims and artifact checks. \\
\bottomrule
\end{tabular}
\end{table}

\noindent
This checklist reflects the repository snapshot date listed in the caption.
Coordination artifacts are maintained as explicit evidence ledgers and updated
when benchmark/code/manuscript states change.



\bibliographystyle{IEEEtran}
\bibliography{references}

\end{document}
