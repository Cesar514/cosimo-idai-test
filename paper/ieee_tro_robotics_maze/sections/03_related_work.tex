\section{Related Work}
Recent navigation literature continues to improve planner quality across classical, hybrid, and learned approaches \cite{S_nchez_Ib_ez_2021,Karur_2021,Xiao_2022,Liu_2023,Abdulsaheb_2023,Hu_2025}. At the same time, multiple surveys report that reproducibility bottlenecks often come from integration variance (runtime stack, simulator configuration, and evaluation protocol) rather than algorithmic novelty alone \cite{Hewawasam_2022,Loganathan_2023}.

For this reason, the closest prior art for our scope is benchmark/runtime infrastructure rather than a single planner family. Benchmark platforms such as Bench-MR \cite{Heiden_2021}, MotionBenchMaker \cite{Chamzas_2022}, and CoBRA \cite{Mayer_2024} focus on reusable benchmark construction and reporting. Runtime frameworks in ROS2/cloud settings emphasize deployment portability and orchestration across heterogeneous environments \cite{Baumann_2021,He_2022,Chen_2024,Shcherbyna_2025}. Our manuscript is positioned at their intersection for one bounded regime: deterministic static grid-maze planner comparison with repository-local traceability.

\begin{table*}[t]
\centering
\caption{Closest infrastructure-oriented prior work and the bounded incremental delta of this paper.}
\label{tab:closest_work_delta}
\footnotesize
\setlength{\tabcolsep}{4.2pt}
\begin{tabular}{p{0.19\textwidth} p{0.31\textwidth} p{0.43\textwidth}}
\toprule
Closest work & Reported primary scope & Incremental delta in this paper \\
\midrule
Bench-MR \cite{Heiden_2021} &
Motion-planning benchmark for wheeled mobile robots. &
Integration of repository-level deterministic seed propagation with trial-level geometric path validation. \\
\midrule
MotionBenchMaker \cite{Chamzas_2022} &
Tooling for generating and benchmarking motion-planning datasets. &
Standardized output normalization across heterogeneous planner schemas coupled with shared-success ranking. \\
\midrule
CoBRA \cite{Mayer_2024} &
Composable benchmark framework for robotics applications. &
Coupled execution stack for planner benchmarking with explicit simulator-fallback and visual regression. \\
\bottomrule
\end{tabular}
\end{table*}


Adjacent systems such as FogROS2-LS and Arena 4.0 provide important deployment/simulation context \cite{Chen_2024,Shcherbyna_2025}, but are not treated as the closest benchmark-protocol comparators in Table~\ref{tab:closest_work_delta}.
The concrete incremental delta claimed in this paper is integration-level: not a new planner, but a deterministic execution stack that couples seed propagation, heterogeneous output normalization, and geometric path validation in one auditable repository workflow.
