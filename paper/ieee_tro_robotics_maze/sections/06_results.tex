\section{Results}
\label{sec:results}

Table~\ref{tab:main_results} summarizes planner performance on 50 generated \(15\times15\) backtracker mazes (seeds 7--56). All 12 planners solved all trials (600/600) with no runtime errors, so differences here primarily reflect computational efficiency rather than reachability.

\begin{table*}[t]
\centering
\caption{Main benchmark results on 50 generated \(15\times15\) backtracker mazes. Rows are ranked by success rate (descending), then comparable mean solve time on shared-success mazes, mean expansions, and overall mean solve time (ascending), with planner name as deterministic tie-break. Time is reported as mean \(\pm\) standard deviation over mazes; median and interquartile range (IQR) are included to expose skew. Lower is better for time, path length, and expansions.}
\label{tab:main_results}
\footnotesize
\setlength{\tabcolsep}{4.2pt}
\begin{tabular}{clccccc}
\toprule
Rank & Planner & Success & Time (ms) $\downarrow$ & Median [IQR] (ms) $\downarrow$ & Path Length $\downarrow$ & Expansions $\downarrow$ \\
\midrule
1  & R1 Weighted A*         & 50/50 & 0.35 $\pm$ 0.21   & 0.29 [0.15, 0.54]   & 142.72          & 187.16  \\
2  & R7 Beam Search         & 50/50 & 0.42 $\pm$ 0.24   & 0.38 [0.20, 0.65]   & 142.72          & 198.36  \\
3  & R5 Jump Point Search   & 50/50 & 0.45 $\pm$ 0.26   & 0.39 [0.19, 0.69]   & 142.72          & 57.26   \\
4  & Greedy Best-First      & 50/50 & 0.46 $\pm$ 0.27   & 0.38 [0.21, 0.71]   & 142.72          & 171.96  \\
5  & R8 Fringe Search       & 50/50 & 0.52 $\pm$ 0.31   & 0.43 [0.23, 0.79]   & 142.72          & 190.30  \\
6  & A*                     & 50/50 & 0.52 $\pm$ 0.31   & 0.43 [0.22, 0.79]   & 142.72          & 189.06  \\
7  & Dijkstra               & 50/50 & 0.54 $\pm$ 0.32   & 0.48 [0.22, 0.84]   & 142.72          & 200.52  \\
8  & R9 Bidirectional BFS   & 50/50 & 0.54 $\pm$ 0.31   & 0.49 [0.24, 0.80]   & 142.72          & 201.52  \\
9  & R2 Bidirectional A*    & 50/50 & 1.25 $\pm$ 0.69   & 1.28 [0.52, 1.92]   & 142.72          & 468.02  \\
10 & R3 Theta*              & 50/50 & 1.55 $\pm$ 0.92   & 1.27 [0.65, 2.35]   & 97.96$^\dagger$ & 189.18  \\
11 & R6 LPA*                & 50/50 & 3.95 $\pm$ 1.59   & 3.56 [2.41, 5.59]   & 142.72          & 295.10  \\
12 & R4 IDA*                & 50/50 & 22.56 $\pm$ 22.13 & 11.05 [2.93, 43.00] & 142.72          & 7061.34 \\
\bottomrule
\end{tabular}

{\raggedright\footnotesize $^\dagger$Theta* uses any-angle motion, so path-length values are not directly comparable to cardinal-grid planners.\par}
\end{table*}

Figures~\ref{fig:benchmark_runtime_ms}, \ref{fig:runtime_uncertainty}, \ref{fig:benchmark_expansions}, and \ref{fig:benchmark_success_rate} provide complementary visual summaries of runtime, runtime uncertainty, search effort, and success rate using the same benchmark snapshot as Table~\ref{tab:main_results}.

\begin{figure}[t]
\centering
\includegraphics[width=\columnwidth]{figures/benchmark_runtime_ms.png}
\caption{Mean planner solve time (ms) on a logarithmic scale over 50 benchmark mazes. Lower values
indicate faster planning.}
\label{fig:benchmark_runtime_ms}
\end{figure}

\begin{figure}[t]
\centering
\includegraphics[width=\columnwidth]{figures/runtime_uncertainty.png}
\caption{Runtime uncertainty over the same 50 paired mazes. Horizontal box summaries are shown on a logarithmic scale (box: IQR, center line: median, red dot: mean).}
\label{fig:runtime_uncertainty}
\end{figure}

\begin{figure}[t]
\centering
\includegraphics[width=\columnwidth]{figures/benchmark_expansions.png}
\caption{Mean node expansions on a logarithmic scale for the same benchmark runs. Lower values
indicate lower search effort.}
\label{fig:benchmark_expansions}
\end{figure}

\begin{figure}[t]
\centering
\includegraphics[width=\columnwidth]{figures/benchmark_success_rate.png}
\caption{Planner success rate over 50 mazes. All methods achieve \(100\%\) in this static benchmark
setting.}
\label{fig:benchmark_success_rate}
\end{figure}

\paragraph{Overall ranking and runtime spread.}
\texttt{r1\_weighted\_astar} is fastest in mean solve time (0.35 ms), followed by \texttt{r7\_beam\_search} (0.42 ms). A second tier---\texttt{r5\_jump\_point\_search}, \texttt{greedy\_best\_first}, \texttt{r8\_fringe\_search}, \texttt{astar}, \texttt{dijkstra}, and \texttt{r9\_bidirectional\_bfs}---is tightly clustered between 0.45 and 0.54 ms (at most +0.19 ms versus the top row). At maze level, \texttt{r1\_weighted\_astar} is fastest on 48/50 mazes, but the best-vs-second-best margin is small (median 0.049 ms; 43/50 mazes within 0.1 ms), indicating limited practical separation among the fastest methods in this setup.
This narrow spread is visible in Figures~\ref{fig:benchmark_runtime_ms} and~\ref{fig:runtime_uncertainty}.

\paragraph{Inferential runtime comparison.}
To characterize runtime consistency across paired mazes, each planner was compared against \texttt{r1\_weighted\_astar} using exact two-sided paired sign tests with Holm correction (family-wise \(\alpha=0.05\)). Effect sizes are reported as paired median runtime deltas (\(\Delta=\text{comparator}-\texttt{r1\_weighted\_astar}\), ms) with 95\% bootstrap confidence intervals from 40{,}000 paired resamples.
\begin{table*}[t]
\centering
\caption{Exploratory paired runtime comparisons against \texttt{r1\_weighted\_astar} on the same 50 mazes (single run per planner-maze pair). Positive \(\Delta\) means the comparator is slower. Confidence intervals are percentile bootstrap intervals from 40{,}000 paired resamples (fixed seed). \(p\)-values are exact two-sided paired sign tests with Holm correction across 11 comparisons.}
\label{tab:runtime_statistical_comparison}
\footnotesize
\setlength{\tabcolsep}{4.0pt}
\begin{tabular}{lcccc}
\toprule
Comparator & Median \(\Delta\) (ms) & 95\% CI for \(\Delta\) (ms) & Slower/Faster (of 50) & Holm-adjusted \(p\) \\
\midrule
A*                       & 0.173  & [0.121, 0.248]         & 50/0 & \(1.95\times10^{-14}\) \\
Dijkstra                 & 0.162  & [0.129, 0.255]         & 50/0 & \(1.95\times10^{-14}\) \\
Greedy Best-First        & 0.094  & [0.064, 0.114]         & 47/3 & \(3.71\times10^{-11}\) \\
R2 Bidirectional A*      & 1.183  & [0.686, 1.540]         & 50/0 & \(1.95\times10^{-14}\) \\
R3 Theta*                & 1.078  & [0.784, 1.747]         & 50/0 & \(1.95\times10^{-14}\) \\
R4 IDA*                  & 14.777 & [6.989, 37.396]        & 50/0 & \(1.95\times10^{-14}\) \\
R5 Jump Point Search     & 0.065  & [0.051, 0.103]         & 50/0 & \(1.95\times10^{-14}\) \\
R6 LPA*                  & 3.609  & [3.037, 4.703]         & 50/0 & \(1.95\times10^{-14}\) \\
R7 Beam Search           & 0.049  & [0.037, 0.070]         & 50/0 & \(1.95\times10^{-14}\) \\
R8 Fringe Search         & 0.174  & [0.122, 0.255]         & 50/0 & \(1.95\times10^{-14}\) \\
R9 Bidirectional BFS     & 0.189  & [0.128, 0.217]         & 50/0 & \(1.95\times10^{-14}\) \\
\bottomrule
\end{tabular}
\end{table*}

For the closest comparator (\texttt{r7\_beam\_search}), the median paired delta is 0.068 ms (95\% CI [0.046, 0.085]) with \texttt{r1\_weighted\_astar} faster on 50/50 mazes; this indicates a consistent but small absolute gain in this dataset. Larger separations appear for slower planners (e.g., \texttt{r6\_lpa\_star}: 3.30 ms [2.73, 4.27]; \texttt{r4\_idastar}: 10.76 ms [5.70, 30.03]). Because each planner-maze pair is measured once, these inferential statistics should be interpreted as exploratory consistency indicators, not hardware-controlled latency certification.

\paragraph{Search-effort trade-offs.}
\texttt{r5\_jump\_point\_search} attains the lowest expansion count (57.26) while remaining in the fast runtime cluster (0.45 ms), reducing expansions by roughly 69.7\% relative to baseline \texttt{astar} (189.06). \texttt{r2\_bidirectional\_astar} and \texttt{r3\_theta\_star} are slower (1.25 and 1.55 ms). \texttt{r6\_lpa\_star} and \texttt{r4\_idastar} are clear outliers at 3.95 ms and 22.56 ms, with \texttt{r4\_idastar} also showing the highest variability (std 22.13 ms; IQR 2.93--43.00 ms) and expansion count (7061.34).
The expansion profile in Figure~\ref{fig:benchmark_expansions} highlights this separation between
the efficient frontier (\texttt{r5\_jump\_point\_search}) and high-overhead outliers.

\paragraph{Uncertainty and limitations.}
These results are limited to one benchmark regime: 50 mazes, one maze size (\(15\times15\)), one generator (backtracker), and one seed schedule. Timings are wall-clock and mostly sub-millisecond for the fastest methods, so statistical significance should not be conflated with large practical effect size in deployment settings. The benchmark also covers static, fully known mazes only; conclusions may not transfer to dynamic, partially observable, or larger-scale environments. Finally, Theta* uses any-angle motion, so its path-length values are not directly comparable to cardinal-grid planners.
The flat success-rate profile in Figure~\ref{fig:benchmark_success_rate} further emphasizes that
this dataset mainly distinguishes planners by efficiency rather than solvability.
