\section{Experimental Setup}
\label{sec:experiments}

This section documents (i) experiments already executed in the current repository snapshot and
(ii) planned studies that are intentionally out of scope for the current results.

\subsection{Executed Experiments}
\textbf{System and workflow context (executed).}
Figure~\ref{fig:system_pipeline} summarizes the deterministic runtime and artifact path used for
the reported experiments, from configuration parsing and planner execution through benchmark export.
Manuscript-process coordination artifacts are tracked in \texttt{coordination/} files and are not
treated as primary scientific evidence in the main experimental narrative.

\begin{figure*}[t]
\centering
\includegraphics[width=\textwidth]{figures/system_pipeline.png}
\caption{Deterministic system pipeline used in the executed experiments, including planner
benchmarking and artifact emission. The runtime path captures backend fallback behavior
(\texttt{pybullet} \(\rightarrow\) \texttt{mujoco} \(\rightarrow\) deterministic fallback) and
the benchmark output branch used for reproducible reporting.}
\label{fig:system_pipeline}
\end{figure*}

\textbf{Planner benchmark (executed).}
We evaluate planners with the repository benchmark harness
(\texttt{robotics\_maze/src/benchmark.py}). The harness generates mazes with fixed settings
(\(50\) mazes, \(15 \times 15\) cells, \texttt{backtracker}, base seed \(7\), per-maze seed
\(= 7 + \texttt{maze\_index}\)) and, by default, fail-closed enforces the canonical 12-planner
evaluation set used in this paper. In the committed benchmark
artifacts (\texttt{robotics\_maze/results/benchmark\_summary.md} and
\texttt{robotics\_maze/results/benchmark\_results.csv}), this corresponds to 12 planners:
\texttt{astar}, \texttt{dijkstra}, \texttt{greedy\_best\_first},
\texttt{r1\_weighted\_astar}, \texttt{r2\_bidirectional\_astar}, \texttt{r3\_theta\_star},
\texttt{r4\_idastar}, \texttt{r5\_jump\_point\_search}, \texttt{r6\_lpa\_star},
\texttt{r7\_beam\_search}, \texttt{r8\_fringe\_search}, and
\texttt{r9\_bidirectional\_bfs}.
The baseline BFS implementation exists in the planner registry but is intentionally excluded from
the benchmark harness default discovered set; all reported rankings and tables use this 12-planner
execution set for consistency with committed artifacts.

Per trial, the harness records success, wall-clock solve time (ms), path length, and node
expansions. Reported paths are validated against occupancy and bounds constraints, including
segment-level collision checks; a trial is marked successful only if both planner output and path
validation succeed. Solve time is measured using Python \texttt{time.perf\_counter()} around each
planner invocation and converted to milliseconds. To reduce first-run cache bias, planner execution
order is rotated per maze. The harness supports configurable repeated-run timing via \texttt{--repeats} \(R\) (number of timed runs per planner-maze pair) and \texttt{--warmup} \(W\) (discarded warm-up invocations before timing begins); when \(R > 1\), per-(planner, maze) aggregate statistics and a Spearman rank-stability report are also emitted. In the current committed artifacts, \(R = 1\) (single measurement), so timing differences should be interpreted as exploratory. Aggregated ranking follows the executable
lexicographic order in \texttt{benchmark.py}: success rate, comparable shared-success solve time,
mean expansions, overall mean solve time, then planner name as deterministic tie-break.
Comparable path length is reported but excluded from ranking because any-angle and cardinal-grid
path geometries are not directly commensurate.

\textbf{Deterministic simulation regression (executed).}
In addition to planner benchmarking, the repository executes deterministic simulation checks
(\texttt{robotics\_maze/testing/run\_sim\_tests.sh}) for three representative planners
(\texttt{astar}, \texttt{weighted\_astar}, \texttt{fringe\_search}) over three episodes
(\texttt{maze-size}=11, \texttt{seed}=42). The same test run invokes the deterministic screenshot
pipeline (\texttt{robotics\_maze/scripts/capture\_regression\_screenshots.py}), which enforces six
expected PNG outputs (three MuJoCo and three fallback renderer images). The latest committed test
log reports all deterministic runs and screenshot checks as PASS.

\begin{table*}[t]
\centering
\caption{Reproducibility protocol, explicitly separated into executed vs planned studies.}
\label{tab:experiment_protocol}
\begin{tabular}{p{0.19\textwidth} p{0.38\textwidth} p{0.38\textwidth}}
\toprule
Component & Executed protocol (current artifacts) & Planned protocol (not yet executed) \\
\midrule
Entrypoints &
\texttt{python robotics\_maze/src/benchmark.py} for planner benchmarks; \texttt{bash robotics\_maze/testing/run\_sim\_tests.sh} for deterministic simulation and screenshot regression. &
Add CI jobs that run full benchmark and regression protocols on pinned hardware classes and publish signed artifacts per run. \\
\midrule
Benchmark workload &
50 mazes, \(15 \times 15\), \texttt{backtracker}, base seed 7, maze seed = \(7 +\) maze index. Current snapshot evaluates 12 planners. &
Scale to larger mazes and multi-generator sweeps (\texttt{backtracker} + \texttt{prim}) with expanded seed ranges to evaluate ranking stability. \\
\midrule
Simulation regression workload &
3 planners (\texttt{astar}, \texttt{weighted\_astar}, \texttt{fringe\_search}), 3 episodes each, \texttt{maze-size}=11, \texttt{seed}=42, backend preference \texttt{auto}. &
Matched paired runs with forced \texttt{pybullet} and \texttt{mujoco}; additional stress scenarios with dynamic obstacles and replanning triggers. \\
\midrule
Primary metrics &
Per trial: success, solve\_time\_ms, path\_length, expansions, and error text on failure. Summary ranking prioritizes success, then comparable shared-success solve time, mean expansions, and overall mean solve time (planner name as deterministic tie-break). Path length is reported descriptively. &
Add variance statistics (median, p95), per-backend deltas, and disturbance-recovery metrics for dynamic scenes. \\
\midrule
Validity checks &
Path-level validation enforces endpoint consistency, in-bounds traversal, and collision-free segments; benchmark marks success only when planner and validator agree. &
Add robustness checks for near-collision margins, execution drift under dynamics, and failure-mode taxonomy by planner family. \\
\midrule
Visual regression &
Deterministic screenshot pipeline requires six fixed filenames (3 MuJoCo + 3 fallback). Latest committed run reports PASS on all checks. &
Add pixel-diff thresholds and automated anomaly triage reports in CI for each benchmark commit. \\
\bottomrule
\end{tabular}
\end{table*}

Table~\ref{tab:experiment_protocol} separates executed protocol components (used for all current
claims) from planned, not-yet-executed studies.

\subsection{Planned Future Experiments (Not Yet Executed)}
The following experiments are planned and are \textbf{not} included in the current result tables:
\begin{enumerate}
    \item \textbf{Cross-algorithm robustness sweep:} expand benchmark runs across
    \texttt{backtracker} and \texttt{prim} generators, larger mazes, and broader seed sets to
    quantify ranking stability.
    \item \textbf{Dynamic-environment stress tests:} introduce moving obstacles and replanning
    triggers to evaluate incremental planners under online disturbances.
    \item \textbf{Post-2021 planner integration study:} add learned local heuristics, Hybrid-A*,
    and uncertainty-aware planning baselines from the repository shortlist, then rerun the same
    benchmark protocol for direct comparability.
    \item \textbf{Backend sensitivity analysis:} run matched scenarios with explicit
    \texttt{pybullet} and \texttt{mujoco} settings to isolate simulator-dependent effects on
    runtime and path execution behavior.
\end{enumerate}
No quantitative claims from these planned experiments are used in the present manuscript version.
